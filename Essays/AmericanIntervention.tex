%----------------------------------------------------------------------------------------
%	PACKAGES AND OTHER DOCUMENT CONFIGURATIONS
%----------------------------------------------------------------------------------------

\usepackage{lastpage} % Required to determine the last page number for the footer

\usepackage{multicol}

\usepackage{nth}

\usepackage{graphicx} % Required to insert images

\setlength\parindent{0pt} % Removes all indentation from paragraphs

\usepackage[most]{tcolorbox} % Required for boxes that split across pages

\usepackage{booktabs} % Required for better horizontal rules in tables

\usepackage{listings} % Required for insertion of code

\usepackage{etoolbox} % Required for if statements

%----------------------------------------------------------------------------------------
%	MARGINS
%----------------------------------------------------------------------------------------

\usepackage{geometry} % Required for adjusting page dimensions and margins

\geometry{
	paper=a4paper, % Change to letterpaper for US letter
	top=3cm, % Top margin
	bottom=3cm, % Bottom margin
	left=2.5cm, % Left margin
	right=2.5cm, % Right margin
	headheight=14pt, % Header height
	footskip=1.4cm, % Space from the bottom margin to the baseline of the footer
	headsep=1.2cm, % Space from the top margin to the baseline of the header
	%showframe, % Uncomment to show how the type block is set on the page
}

%----------------------------------------------------------------------------------------
%	FONT
%----------------------------------------------------------------------------------------

\usepackage[utf8]{inputenc} % Required for inputting international characters
\usepackage[T1]{fontenc} % Output font encoding for international characters

\usepackage[sfdefault,light]{roboto} % Use the Roboto font

%----------------------------------------------------------------------------------------
%	HEADERS AND FOOTERS
%----------------------------------------------------------------------------------------

\usepackage{fancyhdr} % Required for customising headers and footers

\pagestyle{fancy} % Enable custom headers and footers

\lhead{\assignmentTitle} % Left header; output the instructor in brackets if one was set
\chead{} % Centre header
\rhead{\assignmentAuthorName} % Right header; output the author name if one was set, otherwise the due date if that was set

\lfoot{} % Left footer
\cfoot{\small Page\ \thepage\ of\ \pageref{LastPage}} % Centre footer
\rfoot{} % Right footer

\renewcommand\headrulewidth{0.5pt} % Thickness of the header rule

%----------------------------------------------------------------------------------------
%	MODIFY SECTION STYLES
%----------------------------------------------------------------------------------------

\usepackage{titlesec} % Required for modifying sections

%------------------------------------------------
% Section

\titleformat
{\section} % Section type being modified
[block] % Shape type, can be: hang, block, display, runin, leftmargin, rightmargin, drop, wrap, frame
{\Large\bfseries} % Format of the whole section
{\assignmentQuestionName~\thesection} % Format of the section label
{6pt} % Space between the title and label
{} % Code before the label

\titlespacing{\section}{0pt}{0.5\baselineskip}{0.5\baselineskip} % Spacing around section titles, the order is: left, before and after

%------------------------------------------------
% Subsection

\titleformat
{\subsection} % Section type being modified
[block] % Shape type, can be: hang, block, display, runin, leftmargin, rightmargin, drop, wrap, frame
{\itshape} % Format of the whole section
{(\alph{subsection})} % Format of the section label
{4pt} % Space between the title and label
{} % Code before the label

\titlespacing{\subsection}{0pt}{0.5\baselineskip}{0.5\baselineskip} % Spacing around section titles, the order is: left, before and after

\renewcommand\thesubsection{(\alph{subsection})}

%----------------------------------------------------------------------------------------
%	CUSTOM QUESTION COMMANDS/ENVIRONMENTS
%----------------------------------------------------------------------------------------

% Environment to be used for each question in the assignment
\newenvironment{question}{
	\vspace{0.5\baselineskip} % Whitespace before the question
	\section{} % Blank section title (e.g. just Question 2)
	\lfoot{\small\itshape\assignmentQuestionName~\thesection~continued on next page\ldots} % Set the left footer to state the question continues on the next page, this is reset to nothing if it doesn't (below)
}{
	\lfoot{} % Reset the left footer to nothing if the current question does not continue on the next page
}

%------------------------------------------------

% Environment for subquestions, takes 1 argument - the name of the section
\newenvironment{subquestion}[1]{
	\subsection{#1}
}{
}

%------------------------------------------------

% Command to print a question sentence
\newcommand{\questiontext}[1]{
	\textbf{#1}
	\vspace{0.5\baselineskip} % Whitespace afterwards
}

%------------------------------------------------

% Command to print a box that breaks across pages with the question answer
\newcommand{\answer}[1]{
	\begin{tcolorbox}[breakable, enhanced]
		#1
	\end{tcolorbox}
}

%------------------------------------------------

% Command to print a box that breaks across pages with the space for a student to answer
\newcommand{\answerbox}[1]{
	\begin{tcolorbox}[breakable, enhanced]
		\vphantom{L}\vspace{\numexpr #1-1\relax\baselineskip} % \vphantom{L} to provide a typesetting strut with a height for the line, \numexpr to subtract user input by 1 to make it 0-based as this command is
	\end{tcolorbox}
}

%------------------------------------------------

% Command to print an assignment section title to split an assignment into major parts
\newcommand{\assignmentSection}[1]{
	{
		\centering % Centre the section title
		\vspace{2\baselineskip} % Whitespace before the entire section title
		
		\rule{0.8\textwidth}{0.5pt} % Horizontal rule
		
		\vspace{0.75\baselineskip} % Whitespace before the section title
		{\LARGE \MakeUppercase{#1}} % Section title, forced to be uppercase
		
		\rule{0.8\textwidth}{0.5pt} % Horizontal rule
		
		\vspace{\baselineskip} % Whitespace after the entire section title
	}
}

%----------------------------------------------------------------------------------------
%	TITLE PAGE
%----------------------------------------------------------------------------------------

\author{\textbf{\assignmentAuthorName}} % Set the default title page author field
\date{} % Don't use the default title page date field

\title{
	\thispagestyle{empty} % Suppress headers and footers
	\vspace{0.2\textheight} % Whitespace before the title
	\textbf{\assignmentTitle}\\[-4pt]
	\ifdef{\assignmentDueDate}{{\small Due\ on\ \assignmentDueDate}\\}{} % If a due date is supplied, output it
	\ifdef{\assignmentClassInstructor}{{\large \textit{\assignmentClassInstructor}}}{} % If an instructor is supplied, output it
	\vspace{0.32\textheight} % Whitespace before the author name
}


\begin{document}

\begin{center}

  \textbf{America's Secret War against Bolshevism: U.S. Intervention in the Russian Civil War, 1917—1920}$^1$

\end{center}

\begin{justify}

  \hspace{.5in} As the First World War was nearing its end, the Bolsheviks within the Russian Empire began to fight a war of their own. This war, which contributed to the Empire's withdrawal from the First World War, had the intention of overthrowing the Monarchy, and establishing a Marxist-Leninist government. The two parties involved, the \textit{Bolsheviki}, or the pro-Leninist ``reds'' clashed with the anti-Bolsheviks, which consisted of two parties: the pro-Monarchist ``whites'' and the Agrarian-Anarchist ``greens''. Thus, the internal conflict meant that the country in turmoil was focused on itself, resulting in its withdrawal from the First World War with the Treaty of Brest-Litovsk$^2$ (1917), which ceded large portions of land to Germany. Such a turnout worried the Entente, as the effect was two-fold: firstly, Germany did not need to focus on an Eastern front, and, second, Germany had easier access to the port cities of Murmansk and Arkhangelsk; however, the main argument questions whether the motives of the Entente were to actually aid Russia and fend off Germany, or to combat Bolshevism.

  \hspace{.5in} First and foremost, it is important to define the specific focus, as it pertains to time period and location: the referenced article focuses on American intervention in North Russia (specifically Murmansk and Arkhangelsk, not Siberia, which is a different case entirely), from 1918 to 1919, and, consequently, so will this piece. American policy began quite slow, as Woodrow Wilson, then president of the United States, did not want to encourage the Bolsheviks to portray the Entente as enemies who want to take away their Leninist regime$^3$. The inception of American intervention in North Russia began when Wilson, following the October Revolution, established the Supreme War Council (SWC), which was essentially an American military headquarters in Europe. Such a move allowed for faster mobilization and greater influence over policy of the region, as France and Britain were both adamant about getting involved in the Russian conflict$^4$. Albeit intervention would eventually occur, the involvement of Americans in European affairs slowed the process down, as Wilson wanted to find a rational manner in which it would be possible to intervene.

  \hspace{.5in} Though slowed, intervention could not be stopped, and, with the urging of France and Britain, on June 3, 1918, the Supreme War Council passed Joint Note 31$^5$. This note essentially justified American intervention, drawing upon the Entente's ``need'' to protect its allies. This note, however, went against Wilson's wishes, as it anticipated movement from Murmansk and Arkhangelsk into the interior of Russia. Wilson was worried this may lead to a nationalistic uprising, leading to the ideological unification of the Russian peoples in an attempt to purge the western troops. Despite his efforts to appear an ally, Wilson's motives were clear, ``when Wilson agreed to send American soldiers to Archangel, then, he sought not only to conciliate his Allies, but also to help the Russian people liberate their country from an allegedly alien and tyrannical regime.''$^6$ Thus, the view of the Entente with respect to a possible Bolshevik government was clear.

  \hspace{.5in} In an attempt to ease a negative Russian response to intervention, and to clearly define his policy, Wilson released an \textit{aide memoire}$^7$. In it, he stated that the \nth{339} Infantry Regiment would be sent into the port cities ``to guard military stores which may subsequently be needed by Russian forces \ldots''$^8$. Despite all of his attempts to make it appear otherwise, Wilson's public relations campaign was summed up by Foglesong, ``Wilson did want to avoid making the anti-Bolshevik thrust of the military expedition obvious and explicit, but it was implicit from the beginning''$^9$. On September 4, 1918, American troops landed in Arkhangelsk, making the port cities, as stated in Joint Note 31, ``an indispensable corollary of Allied intervention in Siberia''. Again, trying to appear benign, Wilson agreed that Americans would be involved in this effort, so long as there was ``sympathy of the Russian people''$^{10}$. Even with Wilson's attempts, the motives became further evident when American Ambassador David Francis, a staunch anti-Bolshevik, was sent to oversee the operations. While Francis was there, he made it clear that American troops were not idling in Arkhangelsk, and that attempts to move deeper into Russia were occurring. 

  \hspace{.5in} Although they shared a common goal, France and England's overly-aggressive attitude regarding the existence of the Bolsheviks and a possible Bolshevik government pushed Wilson away. This led to an American withdrawal from the port cities — a lengthy withdrawal that lasted until 1920. It is important to remember that, despite this withdrawal, American troops stayed in North Russia well past the conclusion of the First World War, which quite significantly indicates their intentions. The United States' position is most clearly defined in the statement that, ``The Government of the United States has never recognized the Bolshevik authorities and does not consider that its efforts to safeguard supplies at Archangel or to help the Czechs in Siberia have created a state of war with the Bolsheviki''$^{11}$.

  \hspace{.5in} In this manner, it is quite evident that the Entente did not occupy North Russia in an attempt to protect the supplies building up in port cities. Thanks to the placement of David Francis, in addition to France and England's pugnacity, American troops moved southward, against the wishes of Wilson. Even though the goal may have initially been to combat a possible German invasion, it is evident that, as time went on, the intervention became more anti-Bolshevik. Efforts like these, during the inception of the Soviet government, contributed to the East-West divide — tensions which still exist today.

\end{justify}

\newpage

\begin{center}
  
  \textbf{Works Cited:}

\end{center}

\singlespacing

\begin{enumerate}

  \item David S. Foglesong, ``American Intervention in North Russia, 1918-1919.'' America's
    \vspace{-20pt}
    \paragraph{} Secret War against Bolshevism: U.S. Intervention in the Russian Civil War, 1917-
    \vspace{-20pt}
    \paragraph{} 1920 (Raleigh: University of North Carolina Press, 2001), Chapter 7. 

  \item Foglesong, ``American Intervention in North Russia, 1918-1919'' 4.

  \item Foglesong, ``American Intervention in North Russia, 1918-1919'' 5.

  \item Foglesong, ``American Intervention in North Russia, 1918-1919'' 9.

  \item Foglesong, ``American Intervention in North Russia, 1918-1919'' 18.

  \item Foglesong, ``American Intervention in North Russia, 1918-1919'' 11.

  \item Foglesong, ``American Intervention in North Russia, 1918-1919'' 6.

  \item Foglesong, ``American Intervention in North Russia, 1918-1919'' 15.

  \item Foglesong, ``American Intervention in North Russia, 1918-1919'' 3.

  \item Foglesong, ``American Intervention in North Russia, 1918-1919'' 18.

  \item Foglesong, ``American Intervention in North Russia, 1918-1919'' 30.

\end{enumerate}

\end{document}

