%%%%%%%%%%%%%%%%%%%%%%%%%%%%%%%%%%%%%%%%%%%%%%%%%%%%%%%%%%%%%%%%%%%%%%%%%%%%%%%%%%%%%%%%%%%%%%%%%%%%%%%%%%%%%%%%%%%%%%%%%%%%%%%%%%%%%%%%%%%%%%%%%%%%%%%%%%%%%%%%%%%
% Written By Michael Brodskiy
% Class: Honors Inquiry — Twentieth Century Espionage (HONR1310)
% Professor: J. Burds
%%%%%%%%%%%%%%%%%%%%%%%%%%%%%%%%%%%%%%%%%%%%%%%%%%%%%%%%%%%%%%%%%%%%%%%%%%%%%%%%%%%%%%%%%%%%%%%%%%%%%%%%%%%%%%%%%%%%%%%%%%%%%%%%%%%%%%%%%%%%%%%%%%%%%%%%%%%%%%%%%%%

\documentclass[12pt]{article} 
\usepackage{alphalph}
\usepackage[utf8]{inputenc}
\usepackage[russian,english]{babel}
\usepackage{titling}
\usepackage{amsmath}
\usepackage{graphicx}
\usepackage{enumitem}
\usepackage{amssymb}
\usepackage[super]{nth}
\usepackage{everysel}
\usepackage{ragged2e}
\usepackage{geometry}
\usepackage{multicol}
\usepackage{fancyhdr}
\usepackage{cancel}
\usepackage{siunitx}
\usepackage{setspace}

\doublespacing

\geometry{top=1.0in,bottom=1.0in,left=1.0in,right=1.0in}
\newcommand{\subtitle}[1]{%
  \posttitle{%
    \par\end{center}
    \begin{center}\large#1\end{center}
    \vskip0.5em}%

}
\usepackage{hyperref}
\hypersetup{
colorlinks=true,
linkcolor=blue,
filecolor=magenta,      
urlcolor=blue,
citecolor=blue,
}

\pagestyle{fancy}

\lfoot[\vspace{-15pt} \hline]{\vspace{-15pt} \hline}
\rfoot[\vspace{-15pt} \hline]{\vspace{-15pt} \hline}
\cfoot[\thepage]{\thepage}
\chead[\textsc{Twentieth-Century Espionage}]{\textsc{Twentieth-Century Espionage}}
\lhead[\textsc{Paper One}]{\textsc{Paper One}}
\rhead[\textsc{HONR1310}]{\textsc{HONR1310}}



\begin{document}

\begin{center}

  \textbf{America's Secret War against Bolshevism: U.S. Intervention in the Russian Civil War, 1917—1920}$^1$

\end{center}

\begin{justify}

  \hspace{.5in} As the First World War was nearing its end, the Bolsheviks within the Russian Empire began to fight a war of their own. This war, which contributed to the Empire's withdrawal from the First World War, had the intention of overthrowing the Monarchy, and establishing a Marxist-Leninist government. The two parties involved, the \textit{Bolsheviki}, or the pro-Leninist ``Reds'' clashed with the anti-Bolsheviks, which consisted of two parties: the pro-Monarchist ``Whites'' and the Agrarian-Anarchist ``Greens''. Thus, the internal conflict meant that the country in turmoil was focused on itself, resulting in its withdrawal from the First World War with the Treaty of Brest-Litovsk (1917), which ceded large portions of land to Germany.$^2$ Such a turnout worried the Entente, as the effect was two-fold: first, Germany did not need to focus on an Eastern front. Second, Germany had easier access to the port cities of Murmansk and Arkhangelsk. However, the main argument questions whether the motives of the Entente were to actually aid Russia and fend off Germany, or to combat Bolshevism.

  \hspace{.5in} It is important to define the specific focus as it pertains to time period and location. The referenced article focuses on American intervention in North Russia (not Siberia), from 1918 to 1919.Accordingly, American policy began quite slowly, as Woodrow Wilson, then president of the United States, did not want to encourage the Bolsheviks to portray the Entente as enemies who want to take away their Leninist regime.$^3$ The inception of American intervention in North Russia began when Wilson, following the October Revolution, established the Supreme War Council (SWC), which was essentially an American military headquarters in Europe. Such a move allowed for faster mobilization and greater influence over policy of the region, as France and Britain were both adamant about getting involved in the Russian conflict.$^4$ Albeit intervention would eventually occur, the involvement of Americans in European affairs slowed the process down, as Wilson wanted to find a rational manner in which it would be possible to intervene.

  \hspace{.5in} Still, intervention could not be stopped, and with the urging of France and Britain, on June 3, 1918, the Supreme War Council passed Joint Note 31.$^5$ This note essentially justified American intervention, drawing upon the Entente's ``need'' to protect its allies. This note, however, went against Wilson's wishes, as it anticipated movement from Murmansk and Arkhangelsk into the interior of Russia. Wilson was worried this may lead to a nationalistic uprising, leading to the ideological unification of the Russian peoples in an attempt to purge the western troops. Despite his efforts to appear an ally, Wilson's motives were clear — although he did wish to support his allies, his agreement to send troops primarily constituted an effort to undermine the Soviet revolution.$^6$ Thus, the view of the Entente with respect to a possible Bolshevik government was clear.

  \hspace{.5in} In an attempt to ease a negative Russian response to intervention, and to clearly define his policy, Wilson released an \textit{aide memoire}.$^7$ In it, he stated that the \nth{339} Infantry Regiment would be sent into the port cities to protect military stores, which were now vulnerable to German attack.$^8$ Despite all of his attempts to make it appear otherwise, Wilson's public relations campaign essentially wanted to cover up any Allied efforts to meddle in the Soviet revolution, despite implicit attempts to do so.$^9$ On September 4, 1918, American troops landed in Arkhangelsk, making the port cities, as stated in Joint Note 31, ``an indispensable corollary of Allied intervention in Siberia''. Again, trying to appear benign, Wilson agreed that Americans would be involved in this effort, so long as there was ``sympathy of the Russian people''.$^{10}$ Even with Wilson's attempts, the motives became further evident when American Ambassador David Francis, a staunch anti-Bolshevik, was sent to oversee the operations. While Francis was there, he made it clear that American troops were not idling in Arkhangelsk, and that attempts to move deeper into Russia were occurring. 

  \hspace{.5in} Although they shared a common goal, France and England's overly-aggressive attitude regarding the existence of the Bolsheviks and a possible Bolshevik government pushed Wilson away. This led to an American withdrawal from the port cities — a lengthy withdrawal that lasted until 1920. Despite the ordered withdrawal, American troops stayed in North Russia well past the conclusion of the First World War.Thus, the United States' position is most clearly defined in Ambassador Francis's statement, ``The Government of the United States has never recognized the Bolshevik authorities and does not consider that its efforts to safeguard supplies at Archangel or to help the Czechs in Siberia have created a state of war with the Bolsheviki''$^{11}$.

  \hspace{.5in} In this manner, it is quite evident that the Entente did not occupy North Russia in an attempt to protect the supplies building up in port cities. Thanks to the placement of David Francis, in addition to France and England's pugnacity, American troops moved southward, against the wishes of Wilson. Even though the goal may have initially been to combat a possible German invasion, it is evident that, as time went on, the intervention became more anti-Bolshevik. Efforts like these, during the inception of the Soviet government, contributed to the East-West divide — tensions which still exist today.

\end{justify}

\newpage

\begin{center}
  
  \textbf{Works Cited:}

\end{center}

\singlespacing

\begin{enumerate}

  \item David S. Foglesong, ``American Intervention in North Russia, 1918-1919.'' America's
    \vspace{-20pt}
    \paragraph{} Secret War against Bolshevism: U.S. Intervention in the Russian Civil War, 1917-
    \vspace{-20pt}
    \paragraph{} 1920 (Raleigh: University of North Carolina Press, 2001), Chapter 7. 

  \item Foglesong, ``American Intervention in North Russia, 1918-1919'' 4.

  \item Foglesong, ``American Intervention in North Russia, 1918-1919'' 5.

  \item Foglesong, ``American Intervention in North Russia, 1918-1919'' 9.

  \item Foglesong, ``American Intervention in North Russia, 1918-1919'' 18.

  \item Foglesong, ``American Intervention in North Russia, 1918-1919'' 11.

  \item Foglesong, ``American Intervention in North Russia, 1918-1919'' 6.

  \item Foglesong, ``American Intervention in North Russia, 1918-1919'' 15.

  \item Foglesong, ``American Intervention in North Russia, 1918-1919'' 3.

  \item Foglesong, ``American Intervention in North Russia, 1918-1919'' 18.

  \item Foglesong, ``American Intervention in North Russia, 1918-1919'' 30.

\end{enumerate}

\end{document}

