%%%%%%%%%%%%%%%%%%%%%%%%%%%%%%%%%%%%%%%%%%%%%%%%%%%%%%%%%%%%%%%%%%%%%%%%%%%%%%%%%%%%%%%%%%%%%%%%%%%%%%%%%%%%%%%%%%%%%%%%%%%%%%%%%%%%%%%%%%%%%%%%%%%%%%%%%%%%%%%%%%%
% Written By Michael Brodskiy
% Class: Honors Inquiry — Twentieth Century Espionage (HONR1310)
% Professor: J. Burds
%%%%%%%%%%%%%%%%%%%%%%%%%%%%%%%%%%%%%%%%%%%%%%%%%%%%%%%%%%%%%%%%%%%%%%%%%%%%%%%%%%%%%%%%%%%%%%%%%%%%%%%%%%%%%%%%%%%%%%%%%%%%%%%%%%%%%%%%%%%%%%%%%%%%%%%%%%%%%%%%%%%

\documentclass[12pt]{article} 
\usepackage{alphalph}
\usepackage[utf8]{inputenc}
\usepackage[russian,english]{babel}
\usepackage{titling}
\usepackage{amsmath}
\usepackage{graphicx}
\usepackage{enumitem}
\usepackage{amssymb}
\usepackage[super]{nth}
\usepackage{everysel}
\usepackage{ragged2e}
\usepackage{geometry}
\usepackage{multicol}
\usepackage{fancyhdr}
\usepackage{cancel}
\usepackage{siunitx}
\usepackage{setspace}

\doublespacing

\geometry{top=1.0in,bottom=1.0in,left=1.0in,right=1.0in}
\newcommand{\subtitle}[1]{%
  \posttitle{%
    \par\end{center}
    \begin{center}\large#1\end{center}
    \vskip0.5em}%

}
\usepackage{hyperref}
\hypersetup{
colorlinks=true,
linkcolor=blue,
filecolor=magenta,      
urlcolor=blue,
citecolor=blue,
}

\pagestyle{fancy}

\lfoot[\vspace{-15pt} \hline]{\vspace{-15pt} \hline}
\rfoot[\vspace{-15pt} \hline]{\vspace{-15pt} \hline}
\cfoot[\thepage]{\thepage}
\chead[\textsc{Twentieth-Century Espionage}]{\textsc{Twentieth-Century Espionage}}
\lhead[\textsc{Paper One}]{\textsc{Paper One}}
\rhead[\textsc{HONR1310}]{\textsc{HONR1310}}



\begin{document}

\begin{center}

  \textbf{The Origins of French Intervention in the Russian Civil War}$^1$

\end{center}

\begin{justify}

  \hspace{.5in} As 1917 came to a close, and the Entente was clashing with the Central Powers on the Western front, Russia was facing a turmoil of its own. Plunged into a revolutionary civil war following the October Revolution, the Union of Soviet Socialist Republics (formerly the Russian Empire), was engaged in a war of its own. The withdrawal of the Soviet Union from the Eastern Front, a result of the Brest-Litovsk treaty signed in 1917, left a void on the Eastern Front that worried the Entente — would Germany now recoup and send reinforcements from the east to west? This may have been a motive for the French Intervention in Russia, most importantly from January to May of 1918, as most policy changes occurred throughout this period. To this day, the motives are debated. Some scholars, such as A. I. Gukovskii and M. N. Pokrovskii hold that the intentions truly were to prevent consolidation of Central forces; others, such as M. I. Levidov, postulate that this was done to combat the Bolshevik revolution, and prohibit a world-wide revolution of the proletariat.$^{2}$ This view was later exploited by Stalin to propagandize a more unified country threatened by the West. $^{3}$

  \hspace{.5in} As would be expected of the Entente powers who were suspicious of Red influence, France's Foreign Minister at the time, Stephen Pichon, stated that the policy was ``to hasten the end of the anarchy in Russia and to facilitate the reestablishment of order and a legal government.''$^{4}$ First and foremost, this demonstrates that, regardless of the position of the Central Powers, France was set against the possibility of a Communist government. Additionally, by describing the goal as establishing an orderly and legal government, there is no question of whether the French government recognized the legitimacy of the Bolshevik government. Evidently, whether or not the goal later became to combat enemy influence, at some point the goal was to combat the Bolsheviks. In its initial push to undermine Soviet legitimacy, the French Foreign Ministry, the Quai d'Orsay, decided to recognize the Ukrainian government's independence, which harmed the Bolshevik goal of a consolidated government with satellite Soviet-Socialist Republics. Even in doing so, France tried to play both sides, as it never made its statement official to the government of Ukraine.$^{5}$

  \hspace{.5 in} Following the Brest-Litovsk talks in late 1917, the Allied nations began to lay out their respective plans. Britain decided to send R. B. Lockhart to form unofficial ties with the Bolshevik government $^{6}$ and — unbeknownst to most — create a ring of espionage that would result in a large anti-Bolshevik conspiracy. It is at about this point that the $3^{\text{e}}$ Bureau, France's Military Intelligence branch, published a memorandum recommending discussions with the Bolsheviks to form beneficial ties, and prevent the Bolsheviks from connecting with the Central Powers. Despite this, the Quai d'Orsay still opposed any kind of rapprochement.$^{7}$ Eventually, the Chief of the General Staff, Ferdinand Foch, would come to a compromise with Stephen Pichon, which would provide limited support to the Bolsheviks; however, this did not apply to Southern ares of the USSR (a direct reference to the situation in Ukraine).$^{8}$ After reanalysis, the French goal became to promote a sense of Soviet unity, as this would hinder Germany in controlling the territories it had gained through the Brest-Litovsk treaty. The new war directive did improve overall relations with the Bolsheviks, and for some time, Niessel (the General of the mission in Russia) and Trotsky even corresponded regularly. 

  \hspace{.5 in} Once again, however, things began to change, when, on March 20, after becoming the General of the Red Army, Trotsky requested military support. Guillaume Lavergne, Niessel's recent replacement, approved of this, which led to a swift arrival of French military advisors in Russia; however, the Ministry of War also stated that no outside support should be expected, meaning that the French advisors were to use only what they could find in Russia. It is at this point that, despite the Ministry of War's support for limited rapprochement, Joseph Noulens, the French ambassador to Russia, began to outright oppose the Soviet regime, stating that intervention from Arkhangelsk to Vologda and in Siberia should take place, and that any existing agreements should be nullified.$^{9}$ Slowly, opinions toward the Soviet Union would began to shift, as the possibility of a \textit{guerre sociale}, or social war (more accurately, class war), began to rise.$^{10}$ On April 5, the policy slowly began to reverse, as Lavergne was given ``a certain freedom of action'' as a result of a request from the General Staff to the Quai d'Orsay in an attempt to maintain the relations they had built over the last two months. Still, Noulens, as a staunch opponent of rapprochement, requested on April 9 that he be given control of Lavergne's endeavors in Russia. Despite it taking five days for his request to arrive in France, it was approved almost immediately, and he received control on April 16. Thus, in just eleven days, from the fifth to sixteenth of April, the policy which had been started by intelligence and developed by the many French government agencies was reversed.$^{11}$

  \hspace{.5 in} Clearly, many factors played into the oscillating French policy from January to May. It is strongly believed that the policy of rapprochement strongly declined once Foch, a strong supporter, had transferred to the position of commander of Allied armies in France. This is mostly based on the fact that Foch had signed nearly all of the orders supporting collaboration. $^{12}$ In addition to this, the fact that there were so many parties involved in the decision-making regarding the Soviet Union adds to the complexity, as the many different perspectives and opinions on the Bolsheviks are partially responsible for all of the policy shifts. On top of this, the Russian Revolution was historically the first major Bolshevik revolution, which may explain the slow, calculated way in which France tried to deal with its Soviet relations. 

  \hspace{.5 in} Most importantly, the aforementioned questions as to why France intervened are important to discuss. Was it to prevent a consolidation of the Central Powers or prevent a spread of Bolshevik ideals? The correct answer here, as indicated by the evidence, does not lay in an absolute; rather, both were the main goals of French policy at some point, and, thus, both perspectives are partially true. Such are the origins of French intervention in the Russian Civil War.

\end{justify}

\newpage

\begin{center}
  
  \textbf{Works Cited:}

\end{center}

\singlespacing

\begin{justify} 


\end{justify}

\begin{enumerate}

  \item Michael Jabara Carley, “The Origins of the French Intervention in the Russian Civil 
    \vspace{-20pt}
    \paragraph{} War, January-May 1918: A Reappraisal,” \textit{The Journal of Modern History}, Volume
    \vspace{-20pt}
    \paragraph{} 48, Number 3 (September, 1976): 413-439.

  \item Carley, ``The Origins of the French Intervention in the Russian Civil War'' 413-414.
  \item Carley, ``The Origins of the French Intervention in the Russian Civil War'' 414.
  \item Carley, ``The Origins of the French Intervention in the Russian Civil War'' 415.
  \item Carley, ``The Origins of the French Intervention in the Russian Civil War'' 415.
  \item Carley, ``The Origins of the French Intervention in the Russian Civil War'' 416.
  \item Carley, ``The Origins of the French Intervention in the Russian Civil War'' 417.
  \item Carley, ``The Origins of the French Intervention in the Russian Civil War'' 418.
  \item Carley, ``The Origins of the French Intervention in the Russian Civil War'' 421-422.
  \item Carley, ``The Origins of the French Intervention in the Russian Civil War'' 425.
  \item Carley, ``The Origins of the French Intervention in the Russian Civil War'' 426.
  \item Carley, ``The Origins of the French Intervention in the Russian Civil War'' 427.

\end{enumerate}

\end{document}

