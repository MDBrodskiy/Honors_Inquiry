%----------------------------------------------------------------------------------------
%	PACKAGES AND OTHER DOCUMENT CONFIGURATIONS
%----------------------------------------------------------------------------------------

\usepackage{lastpage} % Required to determine the last page number for the footer

\usepackage{multicol}

\usepackage{nth}

\usepackage{graphicx} % Required to insert images

\setlength\parindent{0pt} % Removes all indentation from paragraphs

\usepackage[most]{tcolorbox} % Required for boxes that split across pages

\usepackage{booktabs} % Required for better horizontal rules in tables

\usepackage{listings} % Required for insertion of code

\usepackage{etoolbox} % Required for if statements

%----------------------------------------------------------------------------------------
%	MARGINS
%----------------------------------------------------------------------------------------

\usepackage{geometry} % Required for adjusting page dimensions and margins

\geometry{
	paper=a4paper, % Change to letterpaper for US letter
	top=3cm, % Top margin
	bottom=3cm, % Bottom margin
	left=2.5cm, % Left margin
	right=2.5cm, % Right margin
	headheight=14pt, % Header height
	footskip=1.4cm, % Space from the bottom margin to the baseline of the footer
	headsep=1.2cm, % Space from the top margin to the baseline of the header
	%showframe, % Uncomment to show how the type block is set on the page
}

%----------------------------------------------------------------------------------------
%	FONT
%----------------------------------------------------------------------------------------

\usepackage[utf8]{inputenc} % Required for inputting international characters
\usepackage[T1]{fontenc} % Output font encoding for international characters

\usepackage[sfdefault,light]{roboto} % Use the Roboto font

%----------------------------------------------------------------------------------------
%	HEADERS AND FOOTERS
%----------------------------------------------------------------------------------------

\usepackage{fancyhdr} % Required for customising headers and footers

\pagestyle{fancy} % Enable custom headers and footers

\lhead{\assignmentTitle} % Left header; output the instructor in brackets if one was set
\chead{} % Centre header
\rhead{\assignmentAuthorName} % Right header; output the author name if one was set, otherwise the due date if that was set

\lfoot{} % Left footer
\cfoot{\small Page\ \thepage\ of\ \pageref{LastPage}} % Centre footer
\rfoot{} % Right footer

\renewcommand\headrulewidth{0.5pt} % Thickness of the header rule

%----------------------------------------------------------------------------------------
%	MODIFY SECTION STYLES
%----------------------------------------------------------------------------------------

\usepackage{titlesec} % Required for modifying sections

%------------------------------------------------
% Section

\titleformat
{\section} % Section type being modified
[block] % Shape type, can be: hang, block, display, runin, leftmargin, rightmargin, drop, wrap, frame
{\Large\bfseries} % Format of the whole section
{\assignmentQuestionName~\thesection} % Format of the section label
{6pt} % Space between the title and label
{} % Code before the label

\titlespacing{\section}{0pt}{0.5\baselineskip}{0.5\baselineskip} % Spacing around section titles, the order is: left, before and after

%------------------------------------------------
% Subsection

\titleformat
{\subsection} % Section type being modified
[block] % Shape type, can be: hang, block, display, runin, leftmargin, rightmargin, drop, wrap, frame
{\itshape} % Format of the whole section
{(\alph{subsection})} % Format of the section label
{4pt} % Space between the title and label
{} % Code before the label

\titlespacing{\subsection}{0pt}{0.5\baselineskip}{0.5\baselineskip} % Spacing around section titles, the order is: left, before and after

\renewcommand\thesubsection{(\alph{subsection})}

%----------------------------------------------------------------------------------------
%	CUSTOM QUESTION COMMANDS/ENVIRONMENTS
%----------------------------------------------------------------------------------------

% Environment to be used for each question in the assignment
\newenvironment{question}{
	\vspace{0.5\baselineskip} % Whitespace before the question
	\section{} % Blank section title (e.g. just Question 2)
	\lfoot{\small\itshape\assignmentQuestionName~\thesection~continued on next page\ldots} % Set the left footer to state the question continues on the next page, this is reset to nothing if it doesn't (below)
}{
	\lfoot{} % Reset the left footer to nothing if the current question does not continue on the next page
}

%------------------------------------------------

% Environment for subquestions, takes 1 argument - the name of the section
\newenvironment{subquestion}[1]{
	\subsection{#1}
}{
}

%------------------------------------------------

% Command to print a question sentence
\newcommand{\questiontext}[1]{
	\textbf{#1}
	\vspace{0.5\baselineskip} % Whitespace afterwards
}

%------------------------------------------------

% Command to print a box that breaks across pages with the question answer
\newcommand{\answer}[1]{
	\begin{tcolorbox}[breakable, enhanced]
		#1
	\end{tcolorbox}
}

%------------------------------------------------

% Command to print a box that breaks across pages with the space for a student to answer
\newcommand{\answerbox}[1]{
	\begin{tcolorbox}[breakable, enhanced]
		\vphantom{L}\vspace{\numexpr #1-1\relax\baselineskip} % \vphantom{L} to provide a typesetting strut with a height for the line, \numexpr to subtract user input by 1 to make it 0-based as this command is
	\end{tcolorbox}
}

%------------------------------------------------

% Command to print an assignment section title to split an assignment into major parts
\newcommand{\assignmentSection}[1]{
	{
		\centering % Centre the section title
		\vspace{2\baselineskip} % Whitespace before the entire section title
		
		\rule{0.8\textwidth}{0.5pt} % Horizontal rule
		
		\vspace{0.75\baselineskip} % Whitespace before the section title
		{\LARGE \MakeUppercase{#1}} % Section title, forced to be uppercase
		
		\rule{0.8\textwidth}{0.5pt} % Horizontal rule
		
		\vspace{\baselineskip} % Whitespace after the entire section title
	}
}

%----------------------------------------------------------------------------------------
%	TITLE PAGE
%----------------------------------------------------------------------------------------

\author{\textbf{\assignmentAuthorName}} % Set the default title page author field
\date{} % Don't use the default title page date field

\title{
	\thispagestyle{empty} % Suppress headers and footers
	\vspace{0.2\textheight} % Whitespace before the title
	\textbf{\assignmentTitle}\\[-4pt]
	\ifdef{\assignmentDueDate}{{\small Due\ on\ \assignmentDueDate}\\}{} % If a due date is supplied, output it
	\ifdef{\assignmentClassInstructor}{{\large \textit{\assignmentClassInstructor}}}{} % If an instructor is supplied, output it
	\vspace{0.32\textheight} % Whitespace before the author name
}


\begin{document}

\begin{center}

  \textbf{The Origins of French Intervention in the Russian Civil War}$^1$

\end{center}

\begin{justify}

  \hspace{.5in} As 1917 came to a close, and the Entente was clashing with the Central Powers on the Western front, Russia was facing a turmoil of its own. Plunged into a civil war following the October Revolution, the Union of Soviet Socialist Republics (formerly the Russian Empire), was engaged in a war of its own. The withdrawal of the Soviet Union from the Eastern Front, a result of the Brest-Litovsk treaty signed in 1917, left a void on the Eastern Front that worried the Entente — would Germany now recoup and send reinforcements from the east to west? This may have been a motive for the French Intervention in Russia, most importantly from January to May of 1918, as most policy changes occurred throughout this period. Extensive French espionage networks, which mostly relied on disgruntled Red Army officials and staunch White Army anti-bolsheviks, allowed access to intelligence regarding the state of the Russian Revolution. Despite their facilitated access to intelligence, the amount of players and various agencies involved in French decision-making make their motives, to this day, debated. Some scholars, such as A. I. Gukovskii and M. N. Pokrovskii hold that the intentions truly were to prevent consolidation of Central forces; others, such as M. I. Levidov, postulate that this was done to combat the Bolshevik revolution, and to block further destabilization in Russia in order to restore an Eastern Front.$^{2}$ This view was later exploited by Stalin to propagandize a more unified country threatened by the West. $^{3}$

  \hspace{.5in} Foreign Minister at the time, Stephen Pichon, stated that the policy was ``to hasten the end of the anarchy in Russia and to facilitate the reestablishment of order and a legal government.''$^{4}$ First and foremost, this demonstrates that, regardless of the position of the Central Powers, France was set against establishment of a Communist government. Additionally, by describing the goal as establishing an orderly and legal government, there is no question of whether the French government recognized the legitimacy of the Bolshevik government. Evidently, at some point the goal was to combat the Bolsheviks. In its push to undermine Soviet influence, the French Foreign Ministry, the Quai d'Orsay, recognized the legitimacy of the Ukrainian government, thus undermining the Bolshevik goal of a centralized collection of Soviet Republics. In doing so, France still tried obfuscate their blatant support of a single side, as the statement to recognize Ukraine was never officially published.$^{5}$

  \hspace{.5 in} Following the Brest-Litovsk talks in late 1917, the Allied nations began to lay out their respective plans. Britain decided to send R. B. Lockhart to form unofficial ties with the Bolshevik government $^{6}$ and create a ring of espionage that would result in a large anti-Bolshevik conspiracy. It is at about this point that the $3^{\text{e}}$ Bureau, France's Military Intelligence branch, issued a recommendation to further discussions with the Bolsheviks to form beneficial ties, and prevent the Bolsheviks from connecting with the Central Powers. Despite this, the Quai d'Orsay still opposed any kind of rapprochement.$^{7}$ Eventually, the Chief of the General Staff, Ferdinand Foch, would come to a compromise with Stephen Pichon, which would provide limited support to the Bolsheviks; however, this did not apply to Southern areas of the USSR (a direct reference to the situation in Ukraine).$^{8}$ After reanalysis, the French goal became to promote Soviet solidarity, as this would hinder Germany in controlling the territories it had gained through the Brest-Litovsk treaty. The new war directive did improve overall relations with the Bolsheviks, and for some time, Niessel (the General of the mission in Russia) and Trotsky even corresponded regularly. 

  \hspace{.5 in} On March 20, things began to change once again when, after becoming the General of the Red Army, Trotsky requested military support. Guillaume Lavergne, Niessel's recent replacement, approved of this, which led to a swift arrival of French military advisors in Russia; however, the Ministry of War also stated that no outside support should be expected, meaning that the French advisors were to use only what they could find in Russia. It is at this point that, despite the Ministry of War's support for limited rapprochement, Joseph Noulens, the French ambassador to Russia, began to outright oppose the Soviet regime, stating that intervention from Arkhangelsk to Vologda and in Siberia should take place, and that any existing agreements should be nullified.$^{9}$ Slowly, opinions toward the Soviet Union would begin to shift, as the possibility of a \textit{guerre sociale}, or social war (more accurately, class war), began to rise.$^{10}$ On April 5, the policy slowly began to reverse, as Lavergne was given ``a certain freedom of action'' as a result of a request from the General Staff to the Quai d'Orsay in an attempt to maintain the relations they had built over the last two months. Still, Noulens, as a staunch opponent of rapprochement, requested on April 9 that he be given control of Lavergne's endeavors in Russia. Despite it taking five days for his request to arrive in France, it was approved almost immediately, and he received control on April 16. Thus, in just eleven days, from the fifth to sixteenth of April, the policy which had been started by intelligence and developed by the many French government agencies was reversed.$^{11}$

  \hspace{.5 in} Clearly, many factors played into the oscillating French policy from January to May. It is indicated that the policy of rapprochement lost traction once Foch, a strong supporter, had transferred to the position of commander of Allied armies in France. This is mostly based on the fact that Foch had signed nearly all of the orders supporting collaboration. $^{12}$ In addition to this, the fact that there were so many parties involved in the decision-making regarding the Soviet Union adds to the complexity, as the many different perspectives and opinions on the Bolsheviks are partially responsible for all of the policy shifts. On top of this, the Russian Revolution was historically the first major Bolshevik revolution, which may explain the slow, calculated way in which France tried to deal with its Soviet relations. 

  \hspace{.5 in} Evidently, it is important to question why France intervened. Was it to prevent a consolidation of the Central Powers or prevent a spread of Bolshevik ideals? The correct answer here, as indicated by the evidence, does not lie in an absolute; rather, both were the main goals of French policy at some point, and, thus, both perspectives are partially true. Such are the origins of French intervention in the Russian Civil War.

\end{justify}

\newpage

\begin{center}
  
  \textbf{Works Cited:}

\end{center}

\singlespacing

\begin{justify} 


\end{justify}

\begin{enumerate}

  \item Michael Jabara Carley, “The Origins of the French Intervention in the Russian Civil 
    \vspace{-20pt}
    \paragraph{} War, January-May 1918: A Reappraisal,” \textit{The Journal of Modern History}, Volume
    \vspace{-20pt}
    \paragraph{} 48, Number 3 (September, 1976): 413-439.

  \item Carley, ``The Origins of the French Intervention in the Russian Civil War'' 413-414.
  \item Carley, ``The Origins of the French Intervention in the Russian Civil War'' 414.
  \item Carley, ``The Origins of the French Intervention in the Russian Civil War'' 415.
  \item Carley, ``The Origins of the French Intervention in the Russian Civil War'' 415.
  \item Carley, ``The Origins of the French Intervention in the Russian Civil War'' 416.
  \item Carley, ``The Origins of the French Intervention in the Russian Civil War'' 417.
  \item Carley, ``The Origins of the French Intervention in the Russian Civil War'' 418.
  \item Carley, ``The Origins of the French Intervention in the Russian Civil War'' 421-422.
  \item Carley, ``The Origins of the French Intervention in the Russian Civil War'' 425.
  \item Carley, ``The Origins of the French Intervention in the Russian Civil War'' 426.
  \item Carley, ``The Origins of the French Intervention in the Russian Civil War'' 427.

\end{enumerate}

\end{document}

