\include{Includes.tex}

\title{The Great Game}
\date{\today}
\author{Michael Brodskiy\\ \small Professor: J. Burds}

\begin{document}

\maketitle

\begin{itemize}

    \begin{center}
      \includegraphics{images/AfghanEmir.jpg}
    \end{center}

  \item Conflict between England and Russia for supremacy in Central Asia (1813—1907)

  \item Russia did not have a goal equivalent to ``manifest destiny'', instead they wanted to maintain security

    \begin{itemize}

      \item Stephen Wade claims: ``The core of the Russian expansionist plan was to gradually invade and absorb nations contiguous to its heartland''

    \end{itemize}

  \item Wilhelm Stieber — Chief of Intelligence under Bismarck in Germany

    \begin{itemize}

      \item ``\ldots the first national intelligence chief [in Germany] to use agents to monitor and control the press, the banks, business, and industry''

    \end{itemize}

  \item British fears involved the sepoys: In 1836, there were only 17,000 British troops in India, and of these 1,400 were invalids — ``Common sense dictated that a revolt would be hard to contain''

  \item British views of Russian advances into Central Asia:

    \begin{itemize}

      \item ``That great, grim, shadowy power, which sits brooding over Europe and Asia, and of which no man knows really whether it be strong or weak, whether its people be a young race yet to play a great part in the world's history, or men, as Diderot has described them, as rotten before they are ripe; that dark, silent Russian Czar, the hater of freedom, the foe of every people struggling to cast off oppression. . . . ''

    \end{itemize}

  \item Russian views of British advances into Central Asia:

    \begin{itemize}

      \item ``The position of Russia in Central Asia is that of all civilized states which are brought into contact with half-savage nomadic populations, deficient in any fixed social organization. Under such circumstances, it always happens that the more civilized state is forced, in the interest of the security of its frontier and commercial relations, to exercise a certain ascendancy over those whose turbulent and unsettled character make them undesirable neighbors.''

    \end{itemize}

  \item Bradley Mayhew wrote: “[Secret] Agents posing as scholars, explorers, merchants—even Muslim preachers and Buddhist pilgrims—crisscrossed the mountains, mapping them, spying on each other, courting local rulers, staking claims like dogs in a vacant lot.”

  \item Colin Mazkenzie produced many of the first accurate South Asian maps 

\end{itemize}

\end{document}

