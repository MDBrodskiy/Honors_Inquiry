\include{Includes.tex}

\title{Colonel Alfred Redl — Spy of the Century?}
\date{\today}
\author{Michael Brodskiy\\ \small Professor: J. Burds}

\begin{document}

\maketitle

\begin{itemize}

  \item Habsburg Empire preceding the \nth{20} century

    \begin{itemize}

      \item Franz Joseph became the emperor of the Habsburg Empire in 1848

      \item Major defeats in France and Sardinia

    \end{itemize}

  \item Alfred Redl the person

    \begin{itemize}

      \item Lower class scholarship student, not aristocracy

      \item Very eager to please and ambitious — homosexual favors, informant against fellow soldiers, discretion, advanced him rapidly

      \item Fluent in Polish, Ukrainian, German, and Russian (after spending a year there in 1900)

      \item Idolized by younger officers, although few knew him

      \item Colonel Marchenko ``Clever, reserved, concentrated, and efficient'' versus ``Greedy, but technically perfect Austrian spymaster''

    \end{itemize}

  \item Recruitment by the Russians

    \begin{itemize}

      \item Initial acquaintance with the Russians began sometime in 1901

    \end{itemize}

  \item Redl's Sloppy Style of Espionage

    \begin{itemize}

      \item Exchange of information with Russians

      \item Redl's enormous debts

      \item Extremely lavish lifestyle

      \item Redl's close call to almost being caught

    \end{itemize}

  \item How did Redl affect the Habsburg Empire

    \begin{itemize}

      \item Type of information betrayed to the Russians

      \item Russian counterintelligence gained control of Austria-Hungary's entire offensive and defensive espionage plans

      \item Redl betrayed both agents of the Habsburg Empire and Russian agents

      \item Huge psychological blow to Habsburg Empire

    \end{itemize}

\end{itemize}

\end{document}

