%----------------------------------------------------------------------------------------
%	PACKAGES AND OTHER DOCUMENT CONFIGURATIONS
%----------------------------------------------------------------------------------------

\documentclass[11pt]{article}

%%%%%%%%%%%%%%%%%%%%%%%%%%%%%%%%%%%%%%%%%%%%%%%%%%%%%%%%%%%%%%%%%%%%%%%%%%%%%%%%%%%%%%%%%%%%%%%%%%%%%%%%%%%%%%%%%%%%%%%%%%%%%%%%%%%%%%%%%%%%%%%%%%%%%%%%%%%%%%%%%%%
% Written By Michael Brodskiy
% Class: Honors Inquiry — Twentieth Century Espionage (HONR1310)
% Professor: J. Burds
%%%%%%%%%%%%%%%%%%%%%%%%%%%%%%%%%%%%%%%%%%%%%%%%%%%%%%%%%%%%%%%%%%%%%%%%%%%%%%%%%%%%%%%%%%%%%%%%%%%%%%%%%%%%%%%%%%%%%%%%%%%%%%%%%%%%%%%%%%%%%%%%%%%%%%%%%%%%%%%%%%%

\documentclass[12pt]{article} 
\usepackage{alphalph}
\usepackage[utf8]{inputenc}
\usepackage[russian,english]{babel}
\usepackage{titling}
\usepackage{amsmath}
\usepackage{graphicx}
\usepackage{enumitem}
\usepackage{amssymb}
\usepackage[super]{nth}
\usepackage{everysel}
\usepackage{ragged2e}
\usepackage{geometry}
\usepackage{multicol}
\usepackage{fancyhdr}
\usepackage{cancel}
\usepackage{siunitx}
\usepackage{setspace}

\doublespacing

\geometry{top=1.0in,bottom=1.0in,left=1.0in,right=1.0in}
\newcommand{\subtitle}[1]{%
  \posttitle{%
    \par\end{center}
    \begin{center}\large#1\end{center}
    \vskip0.5em}%

}
\usepackage{hyperref}
\hypersetup{
colorlinks=true,
linkcolor=blue,
filecolor=magenta,      
urlcolor=blue,
citecolor=blue,
}

\pagestyle{fancy}

\lfoot[\vspace{-15pt} \hline]{\vspace{-15pt} \hline}
\rfoot[\vspace{-15pt} \hline]{\vspace{-15pt} \hline}
\cfoot[\thepage]{\thepage}
\chead[\textsc{Twentieth-Century Espionage}]{\textsc{Twentieth-Century Espionage}}
\lhead[\textsc{Paper One}]{\textsc{Paper One}}
\rhead[\textsc{HONR1310}]{\textsc{HONR1310}}

 % Include the file specifying the document structure and custom commands

%----------------------------------------------------------------------------------------
%	ASSIGNMENT INFORMATION
%----------------------------------------------------------------------------------------

% Required
\newcommand{\assignmentQuestionName}{Part} % The word to be used as a prefix to question numbers; example alternatives: Problem, Exercise
\newcommand{\assignmentClass}{} % Course/class
\newcommand{\assignmentTitle}{The Origins of the French Intervention in the Russian Civil War — Handout} % Assignment title or name
\newcommand{\assignmentAuthorName}{Michael Brodskiy} % Student name



\begin{document}


\maketitle % Print the title page

\thispagestyle{empty} % Suppress headers and footers on the title page

\newpage

\begin{multicols}{2}

\begin{center}
  \includegraphics[width=0.95\columnwidth]{images/FirstDays.jpg}
\end{center}
\begin{center}
\includegraphics[width=.475\columnwidth]{images/Clemenceau.jpg}
\includegraphics[width=.475\columnwidth]{images/Foch.png}
\end{center}

\answer{\begin{center}\underline{Major French Players} \end{center}\vspace{-10pt}\begin{itemize} \item Quai d'Orsay — Ministry for Europe and Foreign Affairs \begin{itemize} \vspace{-5pt} \item Philippe Berthelot — Deputy Director \vspace{-5pt} \end{itemize} \item Georges Clemenceau — Prime Minister \vspace{-5pt} \item Henri Albert Niessel — General of the Russian Mission \vspace{-5pt} \item Ferdinand Foch — Chief of General Staff \vspace{-7pt} \item Stephen Pichon — Foreign Minister \vspace{-20pt} \item Joseph Noulens — Ambassador to Russia \end{itemize}}

\end{multicols}

\vspace{-25pt}

\begin{multicols}{2}

  \answer{\begin{itemize} \item Initially, the policy was “to hasten the end of the anarchy in Russia and to facilitate the reestablishment of order and a legal government." — Stephen Pichon \begin{itemize} \item Ukraine was used as a leverage point (by recognizing its independence) \end{itemize} \end{itemize}}
  \vspace{-12pt}
  \answer{\begin{itemize} \item Brest-Litovsk talks, beginning in December 1917, worried the French  \item Military Intelligence suggested rapprochement \item Limited ties with Bolsheviks were approved around mid-February \item Tried to appease Bolsheviks \begin{itemize} \item Ties improved, in early March Niessel corresponded with Trotsky regularly \end{itemize} \end{itemize}}

  \begin{center}
    \includegraphics[width=.95\columnwidth]{images/Ukraine.png}
  \end{center}
  \begin{center}
    \includegraphics[width=.95\columnwidth]{images/Brest-Litovsk.jpeg}
  \end{center}

\end{multicols}

\vspace{-25pt}

\begin{multicols}{2}

  \answer{\begin{itemize} \item On March \nth{20}, after becoming the Military Leader, Trotsky requested military support \vspace{-10pt} \begin{itemize} \item Guillaum Lavergne, who replaced Niessel, approved  \end{itemize} \end{itemize}}

  \answer{\begin{itemize} \item Ministry of War offered limited support to Lavergne \item Noulens asked to reanalyze situation, proposed a break in ties \vspace{3.5pt} \end{itemize}}

\end{multicols}

\assignmentSection{Reversal in Policy}

\vspace{-10pt}

\begin{multicols}{2}

\begin{center}
  \includegraphics[width=.475\columnwidth]{images/Freedom.png}
  \includegraphics[width=.475\columnwidth]{images/Sun.png}
\end{center}

\answer{\begin{itemize} \item Quai d'Orsay began to worry more about spread of \textit{guerre sociale} \item Policy began to reverse, relations deteriorated \item First, on April \nth{5}, Lavergne was ``given a certain freedom of action'' \item Next, on April \nth{9}, Noulens requested control over Lavergne \end{itemize}}

\end{multicols}

\begin{multicols}{2}

  \answer{\begin{itemize} \item Noulens had his request approved on April \nth{16} \item Thus, policy reversal took place from April \nth{5}—\nth{16} \item Foch's resignation played a main role in allowing policy reversal \begin{itemize} \item Historians believe this because he signed most pro-rapprochement documents  \end{itemize} \end{itemize}}

\begin{center}
  \includegraphics[width=.475\columnwidth]{images/Zaem.png}
  \includegraphics[width=.475\columnwidth]{images/Zaem.jpg}
\end{center}

\end{multicols}

\vspace{-40pt}

\assignmentSection{In Conclusion \ldots }

\vspace{-10pt}

\begin{multicols}{2}

  \begin{center}
    \includegraphics[width=.475\columnwidth]{images/Anarchy.jpg} 
    \includegraphics[width=.475\columnwidth]{images/King.jpg} 
  \end{center}

  \answer{\begin{itemize} \item French disorganization contributed to reversal \begin{itemize} \item Many players, with differing perspectives \end{itemize} \item Lots of uncertainty \begin{itemize} \item Marxist revolution never-before-seen \item Many factors to consider in analysis  \end{itemize} \item This makes either historian perspective partially true \end{itemize}}

\end{multicols}

\end{document}
